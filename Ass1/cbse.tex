\documentclass[12pt,-letter paper]{article}
\usepackage{siunitx}
\usepackage{setspace}
\usepackage{gensymb}
\usepackage{xcolor}
\usepackage{caption}
%\usepackage{subcaption}
\doublespacing
\singlespacing
\usepackage[none]{hyphenat}
\usepackage{amssymb}
\usepackage{relsize}
\usepackage[cmex10]{amsmath}
\usepackage{mathtools}
\usepackage{amsmath}
\usepackage{commath}
\usepackage{amsthm}
\interdisplaylinepenalty=2500
%\savesymbol{iint}
\usepackage{txfonts}
%\restoresymbol{TXF}{iint}
\usepackage{wasysym}
\usepackage{amsthm}
\usepackage{mathrsfs}
\usepackage{txfonts}
\let\vec\mathbf{}
\usepackage{stfloats}
\usepackage{float}
\usepackage{cite}
\usepackage{cases}
\usepackage{subfig}
%\usepackage{xtab}
\usepackage{longtable}
\usepackage{multirow}
%\usepackage{algorithm}
\usepackage{amssymb}
%\usepackage{algpseudocode}
\usepackage{enumitem}
\usepackage{mathtools}
%\usepackage{eenrc}
%\usepackage[framemethod=tikz]{mdframed}
\usepackage{listings}
%\usepackage{listings}
\usepackage[latin1]{inputenc}
%%\usepackage{color}{   
%%\usepackage{lscape}
\usepackage{textcomp}
\usepackage{titling}
\usepackage{hyperref}
%\usepackage{fulbigskip}   
\usepackage{tikz}
\usepackage{graphicx}
\lstset{
  frame=single,
  breaklines=true
}
\let\vec\mathbf{}
\usepackage{enumitem}
\usepackage{graphicx}
\usepackage{siunitx}
\let\vec\mathbf{}
\usepackage{enumitem}
\usepackage{graphicx}
\usepackage{enumitem}
\usepackage{tfrupee}
\usepackage{amsmath}
\usepackage{amssymb}
\usepackage{mwe} % for blindtext and example-image-a in example
\usepackage{wrapfig}
\graphicspath{{figs/}}
\providecommand{\mydet}[1]{\ensuremath{\begin{vmatrix}#1\end{vmatrix}}}
\providecommand{\myvec}[1]{\ensuremath{\begin{bmatrix}#1\end{bmatrix}}}
\providecommand{\cbrak}[1]{\ensuremath{\left\{#1\right\}}}
\providecommand{\brak}[1]{\ensuremath{\left(#1\right)}}
\begin{document}
\section*{\centering SECTION-A}
\begin{enumerate}
	\item Write the coordinates of the point which is the reflection of the point \brak {\alpha, \beta, \gamma} in the XZ-plane.
    
  \item  Find the position vector of the point which divides the join of points with position vectors $\vec{a}+3\vec{b}$ and $\vec{a}-\vec{b}$ internally in the ratio $1:3$.
    
  \item If $\mydet{\vec{a}} = 4 , \mydet{\vec{b}} = 3$ ,and $\vec{a} \cdot \vec{b}= 6\sqrt{3} $,then find the value of $\mydet{\vec{a} \times \vec{b}}$.
    
    \item Write the value of $\mydet{a-b&b-c&c-a\\b-c&c-a&a-b\\c-a&a-b&b-c}$.
    \item  If $ \vec{A} =\myvec{ 1 & -2 & 3 \\-4 & 2 & 5}$ and $ \vec{B}=\myvec{2&3\\4&5\\2&1}$ and  $BA$=$\brak{b_{ij}}$,find $b_{21}+b_{32}$.
    \item Write the number of all possible matrices of order $2 \times 3$  with each entry $1$ or $2$.

\section*{\centering SECTION-B}    
  \item  Find the equation of the tangent line to the curve $ y = \sqrt{5x - 3} - 5 $, which is parallel to the line $ 4x - 2y + 5 = 0 $.

   \item Solve the differential equation:\\$\brak{x^2+3xy+y^2}dx-x^2dy=0$ given that $y=0$, when $x=1$.

    \item On her birthday Seema decided to donate some money to the children of an orphanage home. If there were $8$ children less, every one would have got \rupee 10 more. However, if there were $16$ children more, every one would have got \rupee 10 less. Using the matrix method, find the number of children and the amount distributed by Seema. What values are reflected by Seema's decision?

    \item Show that the lines $\frac {x-1} {3} $ = $\frac {y-1} {-1} $ = $\frac {z+1} {0} $ and $\frac {x-4} {2} $ = $\frac {y} {0} $ = $\frac {z+1} {3} $ intersect.\\Find their point of intersection.
	
     \item Show that the function $f$ given by: \\
f(x)=$\begin{cases}\frac{e^{\frac{1}{x}}-1}{e^{\frac{1}{x}}+1}, if x\neq0\\    {-1}, if x = 0 \end{cases}$ \\is discontinuous at $x$ = $0$.

\item Find:$\int  \frac {2x+1} {\brak{x^2+1} \brak{x^2+4}}$dx

     \item If $ x = e^{\cos2t} and   y = e^{\sin2t} $,prove that $ \frac {dy} {dx} $ = $ -\frac{y \log x} {x \log y}	$. 
     \item Verify Mean Value theorem for the function $f\brak{x}=2\sin x+\sin 2x$ on $\myvec{0,\pi} $.
	

     \item Solve for x: $\tan^{-1} \brak{\frac {2 - x}{2 + x}} = \frac {1}{2} \tan^{-1}\brak{\frac {x}{2}}$,$x > 0$. 
      \item Prove that $2sin^{-1}\brak{\frac{3}{5}}-tan^{-1}\brak{\frac{17}{31}}$=$\frac{\pi}{4}$.

      \item Evaluate: $\int_{1}^{5}$ $\{{\mydet{x - 1} + \mydet{x - 2} + \mydet{x - 3}}\}$dx 
      \item Evaluate:$\int_{0}^{\pi}$ $ \frac {xsinx} {1+3cos^2x} $dx	     

      \item  $x\frac {dy} {dx}  + y - x + xycotx=0$; $x\neq0$.
	      
      \item  A committee of $4$ students is selected at random from a group consisting of $7$ boys and $4$ girls. Find the probability that there are exactly $2$ boys in the committee, given that at least one girl must be there in the committee.

      \item Find the angle between the vector $\vec{a}+\vec{b}$ and $\vec{a}-\vec{b}$ if $\vec{a}$=$2\hat{i}-\hat{j}+3\hat{k}$, and $\vec{b}$=$3\hat{i}+\hat{j}-2\hat{k}$ hence find a vector perpendicular to both $\vec{a}+\vec{b}$ and $\vec{a}-\vec{b}$

      \item Find: $\int \brak{3x+5}\sqrt {5+4x-2x^2}$ dx

	      \section*{\centering SECTION-C}	      

      \item Using integration, find the area of the triangle formed by negative x-axis and tangent and normal to the circle $x ^ 2 + y ^ 2 = 9$ at $\brak{- 1, 2\sqrt2}$ .  					     
      \item A company manufactures two types of cardigans: type A and type B. It costs \rupee 360 to make a type A cardigan and \rupee 120 to make a type B cardigan. The company can make at most $300$ cardigans and spend at most \rupee 72,000 a day. The number of cardigans of type B cannot exceed the number of cardigans of type A by more than 200. The company makes a profit of \rupee 100 for each cardigan of type A and \rupee 50 for every cardigan of type B.\\Formulate this problem as a linear programming problem to maximise the profit to the company. Solve it graphically and find maximum profit.

       \item Find the coordinates of the foot of perpendicular and perpendicular distance from the point $P\brak{4, 3, 2}$ to the plane $x + 2y + 3z = 2$ Also find the image of $P$ in the plane.  
       \item  Solve for x: $\mydet{a+x&a-x&a-x\\ a-x&a+x&a-x\\a-x&a-x&a+x}$=$0$,using properties of determinants.
       \item  Using elementary row operation find the inverse of matrix  $A$=$\myvec{3&-3&4\\2&-3&4\\0&-1&1}$ and hence solve the following system of equations $3x-3y+4z=21,2x-3y+4z=20,-y+z=5$.                                                 
        \item $A$, $B$ and $C$ throw a pair of dice in that order alternately till one of them gets a total of $9$ and wins the game. Find their respective probabilities of winning, if $A$ starts first.
       
	\item Show that the relation $R$ defined by \brak{a,b} R \brak{c,d} $\Rightarrow a+d=b+c $ on the $A \times A$, where $A=\{1,2,3........,10\}$ is an equivalence relation.Hence write the equivalence class $\myvec{\brak{3,4}};a,b,c,d \epsilon A$.

       \item Show that height of the cylinder of greatest volume which can be inscribed in a right circular cone of height $h$ and semi-vertical angle $\alpha$ is one-third that of  the cone and the greatest volume of the cylinder is $\frac {4} {27} \pi h^3 \tan^2 \alpha $. 
        \item Find the intervals in which the function  f\brak{x}=$ \frac {4sinx} {2+cosx} $ - x;$0 \leq x \leq 2 \pi$ is strictly incresing or strictly decreasing  
              
\end{enumerate}
\end{document}                                                 
